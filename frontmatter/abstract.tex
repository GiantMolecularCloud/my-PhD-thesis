%*******************************************************
% Abstract
%*******************************************************

\chapter*{Abstract}
\chaptermark{abstract}

\renewcommand{\abstractname}{Abstract}
\pdfbookmark[1]{Abstract}{Abstract}
\addcontentsline{toc}{chapter}{\tocEntry{Abstract}}

\begingroup
\let\clearpage\relax
\let\cleardoublepage\relax
\let\cleardoublepage\relax


%%%%%%%%%%%%%%%%%%%%%%%%%%%%%%%%%%%%%%%%%%%%%%%%%%%%%%%%%%%%%%%%%%%%%%%%%%%%%%%%%%%%%%%%%%%%%%%%%%%%

Starburst galaxies are characterized by intense star formation at high star formation rate surface densities and short gas depletion times. 
Strong stellar feedback drives galaxy-scale winds and outflows in all gas phases, i.e. ionized, neutral and molecular.
The extreme conditions in local starburst galaxies are thought to be similar to those in typical high-redshift star forming galaxies, e.g. at the peak of the cosmic star formation history.

In this thesis, we present and analyze 0.15\arcsec ($\sim2.5$\,pc) resolution ALMA \co32 observations of the nuclear starburst in \ngc253. 
Using this data, we study the molecular outflow in unprecedented detail, zoom into \ngc253's super star clusters (SSCs) and compare the starburst to the similar but more quiescent center of the Milky Way.

Firstly, we kinematically decompose the molecular gas emission in \ngc253 into a disk and non--disk component to then separate out the molecular outflow. We systematically improve on previous measurement and obtain mass outflow rates $\dot{M} = 14-39$\,\Msunyr for the starburst. %($SFR \sim 2$\,\Msunyr).
The kinetic energy and momentum of the molecular outflow dominates over the other gas phases and is consistent with being supplied by the starburst at a few percent efficiency.

Secondly, we study the physical and chemical conditions in the molecular gas in the (proto-)SSCs in \ngc253, the places where future outflows will be launched from. The SSCs differ significantly in chemical complexity and show up to 55 lines belonging to 14 different chemical species. Spectral modelling allows us to infer spectral line ratios and physical properties.
The molecular gas in the SSCs is hot, consistent with UV photon-dominated chemistry and permeated by intense infrared radiation.

Thirdly, we compare the molecular cloud properties in the starbursting center of \ngc253 and the Milky Way Galactic Center (GC), that shares similar properties as \ngc253.
Using a structure identification algorithm on resolution-, area- and noise-matched datasets allows for a direct comparison of the kinematic structure.
Through common cloud scaling relations, we infer a high external pressure ($P_\mathrm{ext} \sim 10^{7-7.5}$\,K\,\pcm3) in \ngc253 and a significant amount of unbound (non-self-gravitating) molecular gas that is characterized by high velocity dispersion.

In summary, in this thesis we could follow the life cycle of a molecular outflow from an actively star forming molecular cloud before the launching of an outflow all the way out to distances that are hundreds of parsecs above the starburst disk where the outflow fades away.


%%%%%%%%%%%%%%%%%%%%%%%%%%%%%%%%%%%%%%%%%%%%%%%%%%%%%%%%%%%%%%%%%%%%%%%%%%%%%%%%%%%%%%%%%%%%%%%%%%%%

% \vfill
\newpage

\begin{otherlanguage}{ngerman}

\chapter*{Zusammenfassung}
\pdfbookmark[1]{Zusammenfassung}{Zusammenfassung}

%%%%%%%%%%%%%%%%%%%%%%%%%%%%%%%%%%%%%%%%%%%%%%%%%%%%%%%%%%%%%%%%%%%%%%%%%%%%%%%%%%%%%%%%%%%%%%%%%%%%

Starburst-Galaxien sind charakterisiert durch intensive Sternentstehung mit hohen Flächendichten der  Sternentstehungsrate und schnellem Verbrauch ihres Gases.
Starke stellare Rückkopplungsmechanismen erzeugen Winde und Ausflüsse auf galaktischen Skalen in allen Gasphasen, d. h. ionisiert, neutral and molekular.
Es wird davon ausgegangen, dass die extremen Bedingungen in lokalen Starburst-Galaxien ähnlich denen in typischen hochrotverschobenen Galaxien mit aktiver Sternentstehung sind. 

In dieser Arbeit präsentieren und analysieren wir ALMA Beobachtungen der nuklearen Starburst-Galaxie \ngc253 mit 0.15\arcsec ($\sim 2.5$\,pc) Auflösung.
Mithilfe dieser Daten studieren wir den molekularen Ausfluss in nie dagewesenem Detailreichtum, zoomen in die Supersterncluster (SSCs) und vergleichen den Starburst mit dem ähnlichen, aber ruhigeren Zentrum der Milchstraße.

Wir zerlegen die beobachtete Emission des molekularen Gases in \ngc253 in Scheiben- und Nicht-Scheiben-Komponenten, um daraufhin den molekularen Ausfluss zu identifizieren.
Wir verbessern systematisch vorherige Messungen und erhalten Ausflussmassenraten $\dot{M} = 14-39$\,\Msunyr für den Starburst.
Die kinetische Energie und der Impuls des molekularen Ausflusses dominieren über die anderen Gasphasen.
Energie und Impuls werden vom Starburst erzeugt und mit wenigen Prozent Effzienz an das molekulare Gas übertragen.

Desweiteren untersuchen wir die physikalischen und chemischen Bedingungen im molekularen Gas der (proto-)SSCs in \ngc253, der Orte an denen zukünftige Ausflüsse ausgeschleudert werden.
Die SSCs unterscheiden sich signifikant in ihrer chemischen Komplexität und zeigen bis zu 55 Spektrallinien von 14 verschiedenen Molekülen.
Durch Modellierung der Spektren erhalten wir Linienverhältnisse und physikalische Eigenschaften.
Das molekulare Gas in den SSCs ist heiß, konsistent mit Chemie dominiert von UV Photonen und durchdrungen von intensiver infraroter Strahlung.

Zuletzt vergleichen wir die Eigenschaften der molekularen Wolken im Starburst im Zentrum von \ngc253 mit dem Galaktischen Zentrum der Milchstraße, das ähliche Eigenschaften wie \ngc253 aufweist.
Mithilfe eines Algorithmus zur Strukturidentifikation, angewendet auf Daten angeglichener Auflösung, Abdeckung und Rauschens, wird die kinematische Struktur verglichen.
Skalierungsbeziehungen der Wolken zeigen hohen externen Druck ($P_\mathrm{ext} \sim 10^{7-7.5}$\,K\,\pcm3) auf die Wolken und signifikante Mengen an ungebundenem (nicht selbst-gravitierendem) molekularem Gas, das durch hohe Geschwindigkeitsdispersion charakterisiert wird.

Zusammenfassend folgt diese Arbeit dem Leben eines molekularen Ausflusses von einer aktiv sternbildenden molekularen Wolke vor Auswurf eines Ausflusses bis hin zu hunderten Parsec über dem Starburst, wo der Ausfluss verblasst.


%%%%%%%%%%%%%%%%%%%%%%%%%%%%%%%%%%%%%%%%%%%%%%%%%%%%%%%%%%%%%%%%%%%%%%%%%%%%%%%%%%%%%%%%%%%%%%%%%%%%

\end{otherlanguage}

\endgroup

\vfill
