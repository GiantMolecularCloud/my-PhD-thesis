%*******************************************************
% Cover page backside
%*******************************************************
\thispagestyle{empty}
%\pdfbookmark[1]{Titel}{title}
%*******************************************************

% \newgeometry{centering}    %%% make the page centered on paper

% \begin{center}
%     {\large \color{CTtitle}
%         \spacedallcaps{\myTitle}
%     }
% \end{center}    

% \vfill

% \hspace{2cm}    
% \begin{tabular}{ll}
%     {\color{CTtitle} \spacedallcaps{Referees:}} & {\large \myExaminerOne}\\
%                                                 & {\large \myExaminerTwo}\\
% \end{tabular}


% \restoregeometry

\hfill
\vfill
\noindent

\textit{\textbf{Cover design:} 
The molecular gas in \ngc253 traced by carbon monoxide (CO). The three-color composite shows CO gas at the systemic velocity in green, gas approaching the observer in blue and receding gas in red. The rotating disk of molecular gas produces a color gradient (red -- green -- blue) from top left to bottom right. Seemingly mismatching structures all along the lower left and upper right edges of the disk highlight the abundant molecular streamers that are driven out of the galaxy by stellar feedback.
}

\vspace{1cm}
\textit{\textbf{Flipbook:}
This thesis is in large parts based on a dataset obtained with the ALMA interferometer. Such datasets contain three dimensional information (two spatial, one velocity dimensions) that are impossible to present in a printed document. Instead, images in the bottom left and right corner show the spatial information in succession of increasing velocity. When quickly scrolled through as a flipbook, the images show the rotating molecular gas disk in the center of \ngc253 similar to the movie function in common data visualization tools.
}