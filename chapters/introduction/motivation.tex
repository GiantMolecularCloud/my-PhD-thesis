%%%%%%%%%%%%%%%%%%%%%%%%%%%%%%%%%%%%%%%%%%%%%%%%%%%%%%%%%%%%%%%%%%%%%%%%%%%%%%%%%%%%%%%%%%%%%%%%%%%%

\chapter{Motivation}
\chaptermark{motivation}
\label{part: motivation}

%%%%%%%%%%%%%%%%%%%%%%%%%%%%%%%%%%%%%%%%%%%%%%%%%%%%%%%%%%%%%%%%%%%%%%%%%%%%%%%%%%%%%%%%%%%%%%%%%%%%

% Motivation is what's needed to get out of bed in the morning.\\
% \hspace*{\fill} --- me.
% \vspace{1cm}

Understanding our place in the universe and how the universe came to be is one of the most basic existential desires of humankind -- probably since humans evolved far enough to develop such complex thoughts. The history of the universe is complicated and long, almost 14 billion years by now. We have learned much through astronomy and astrophysical research came closer to answers to these basic questions. However, many fundamental questions are still open. For instance, how and why did we end up on the blue planet thinking about all of this?
\vspace{0.25\baselineskip}

Exploring where we come from and how we came to be touches on the fields of cosmology, astrophysics, geology, biology and more. After the big bang and the early phases of the universe, galaxies of dark matter, gas and stars started to assemble. Through merging and accretion of primordial gas, these galaxies built up their stellar content with a whole system of planets for each star. The cosmic star formation rate accelerated quickly in the early universe, slowed down and eventually started to decline after two to three billion years. This means many if not most stars in our own Milky Way and other local galaxies formed billions of years ago when the conditions were different from today.

Even for down-to-earth questions such as how intelligent life could develop on earth, it is therefore necessary to understand star formation in general and in particular how star formation worked at the time the sun and our planetary system formed about five billion years ago. Traces of how the sun formed are long gone but thanks to the finite speed of light, we are able to observe past times when we observe objects far away from us.
Studying star forming galaxies that are billions of lightyears away from us in the required detail is not possible with today's technology but luckily there are local galaxies that are very similar in their properties. The mentioned decline of the star formation rate history means that what used to be a normal star forming galaxy like all the others billions of years ago, today, is an exceptionally intensely star forming -- star bursting -- galaxy.

Many details of starbursts are not yet understood. Star formation is a complex process covering spatial scales from galaxy sizes down to single stars over many orders of magnitude. The gas and dust in the interstellar medium from which stars form is already a thermodynamically complex mixture of gases, solids and plasmas. The picture gets even more complex when considering the interaction between (forming) stars and the interstellar medium. This feedback can have diametral effects in shutting down star formation or enhancing it depending on the details of the exact situation. Gaseous outflows driven by feedback from stars or actively accreting black holes are an integral part in determining the evolution of a galaxy.

In this thesis, I work on certain aspects of star formation, stellar feedback and galactic outflows to deepen humankind's understanding of star formation and the evolution of the universe.
I characterize outflows of molecular gas, zoom into the environment of forming super star clusters and examine the feedback effects on dense molecular gas in galactic centers.
Each of these works is only a tiny piece in the grand puzzle that is the universe, but finding each piece and putting it in the right location is necessary to build up the full picture of the universe and the life within it.


\newpage
\begin{overpic}[width=\textwidth]{images/chapters/introduction/ism/overview.pdf}
    \label{introduction: figure: topics overview}
    \put (5,90) {\color{white} \scalebox{2}{What this thesis addresses:}}
    \put (5,82) {\color{white} \Large \textbf{interstellar medium}}
        \put (5,78) {\color{white} \large Chapters \ref{introduction: chapter: ism}, \ref{chapter: outflow}, \ref{chapter: outflow catalog}, \ref{chapter: SSCs}, \ref{chapter: dendro}}
    \put (38,76) {\color{white} \Large \textbf{star formation}}
        \put (38,72) {\color{white} \large Chapters \ref{introduction: chapter: star formation}, \ref{chapter: outflow}, \ref{chapter: outflow catalog}, \ref{chapter: SSCs}, \ref{chapter: dendro}}
    \put (8,65) {\color{white} \Large molecular gas}
        \put (8, 61){\color{white}\large Chapters \ref{introduction: chapter: ism}, \ref{introduction: chapter: star formation}, \ref{chapter: outflow}, \ref{chapter: outflow catalog}, \ref{chapter: SSCs}, \ref{chapter: dendro}}
    \put (7,23) {\color{white} \Large star formation fueling}
        \put (7,19) {\color{white} \large Chapters \ref{introduction: chapter: star formation}, \ref{chapter: SSCs}}
    \put (36,33) {\color{white} \Large embedded (proto-)}
    \put (36,30) {\color{white} \Large super star cluster}
        \put (36,26) {\color{white} \large Chapters \ref{introduction: section: star formation: SSCs}, \ref{chapter: SSCs}}
    \put (70,49) {\color{white} \Large stellar feedback}
        \put (70,45) {\color{white} \large Chapters \ref{introduction: chapter: star formation}, \ref{chapter: outflow}, \ref{chapter: outflow catalog}, \ref{chapter: SSCs}, \ref{chapter: dendro}}
    \put (52,20) {\color{white} \Large outflow driving}
        \put (52,16) {\color{white} \large Chapters \ref{chapter: outflow}, \ref{chapter: outflow catalog}}
    \put (72,73) {\color{white} \Large cloud destruction}
        \put (81.5,69) {\color{white} \large Chapters \ref{introduction: chapter: star formation}, \ref{chapter: SSCs}}
    \put (5,7) {\color{white} \Large The life cycle of a molecular outflow from prior to}
    \put (5,3) {\color{white} \Large launching by stellar feedback to fading far above the galactic disk.}
\end{overpic}

\vspace{0.5cm}
\begin{tabular}{@{}ll}
    \textit{Images:} & Super star cluster Westerlund~1, HST/NASA/ESA, HST ID: \href{https://www.spacetelescope.org/images/potw1710a/}{potw1710a}\\
     & Milky Way panorama, ESO/S. Brunier, ESO ID: \href{https://www.eso.org/public/images/eso0932a/}{eso0932a}\\
    \textit{Composition:} & N. Krieger
\end{tabular}
